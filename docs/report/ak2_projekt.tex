\documentclass[11pt]{report}

\usepackage[T1]{fontenc}
\usepackage[polish]{babel}
\usepackage{mathptmx}
\usepackage[utf8]{inputenc}
\usepackage{lmodern}
\usepackage{enumitem}
\usepackage{setspace}
\usepackage{advdate}
\usepackage[none]{hyphenat}
\selectlanguage{polish}
\usepackage[left=2.5cm,top=2.5cm,right=2.5cm,bottom=2.5cm,bindingoffset=0.5cm]{geometry}



\title{SO2 Temat Projektu}
\date{}
\author{Jakub Sokołowski}
\newcommand{\LabNum}{1}
\setcounter{chapter}{\LabNum}
\begin{document}
\noindent
Jakub Sokołowski 226080, Mariola Wróbel 226112 \hfill Wrocław, dn.\ {\AdvanceDate[-1]\today}\\

\noindent
ŚR-N-11\hfill prowadzący: dr inż. Dominik Żelazny
\vspace{1cm}
\begin{center}
  \begin{Large}
  	Projekt z Architektur Komputerów 2\\
    \emph{Kodowanie danych za pomocą C i asemblera}
  \end{Large}
\end{center}
\section{Opis projektu}
Zaszyfrowana komunikacja między procesami. Wysyłanie zaszyfrowanej wiadomości z jednego procesu do drugiego przy użyciu różnych protokołów i algorytmów kryptograficznych. Algorytmy zostaną zaimplementowane przy użyciu języka C i asemblera.
\section {Etapy projektu}
\begin{itemize}
\item{I Etap: \\Stworzenie dwóch procesów i przekazanie pomiędzy nimi zakodowanej za pomocą \textbf{base64} wiadomości. Komunikacja między procesami będzie odbywała sie w warstwie C, z użyciem funkcji fork(), a alogrytmy kodujące i szyfrujące będą napisane w asemblerze.}
\item{II Etap: \\Zaszyfrowanie wiadomości za pomocą szyfru blokowego. Przekazanie wiadomości i klucza do drugiego procesu i odszyfrowanie wiadomości. Dodanie możliwości przesyłania plików.}
\item{III Etap: \\Ustalenie wspólnego klucza za pomocą wymiany \textbf{Diffieg-Hellman}. Bezpieczna komunikacja między procesami. Użycie bibloteki \textbf{OpenSSL}, do pobrania dużych liczb pierwszych i~generowanie kryptograficznie bezpiecznych liczb losowych.}

\end{document}